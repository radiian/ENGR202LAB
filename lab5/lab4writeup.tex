%Writeup for ENGR202 Lab {INSERT LAB #}
%If editing in vim/vi please insert your own line breaks! This will help keep the file cleaner and easier to edit.

%::Collaboration etiquette::
%In order to better identify changes
%please surround all edits with your
%name as follows:

%BEGIN Zach
%END Zach

%LaTeX will ignore line breaks so to denote
%changes in the middle of a line begin the
%edits on a new line after your name tag
%and then continue with the original on
%the next line following your ending name
%tag.



\documentclass{article}
\usepackage{graphicx}
\graphicspath{ {images/} }

%Change # to correct lab number!
\title{Chapter 5 Lab Writeup}
\author{Zach Thompson, Simon Hannes, Kyle Peterson}
\begin{document}

\maketitle{}

\section*{Overview}
\paragraph{}
%BEGIN ZACH
The purpose of this lab was to examine different types of filters and how they react to changing frequencies.
Three different filters were given to classify.

%END ZACH


\section*{Process}
%BEGIN KYLE
\paragraph{}
To determine the filter type of each circuit we are taking two measurements on the osciloscope at each of the 
10 frequencies in the table. The voltage is measured across the capacitor, inductor, and resistor for the 
respective mystery filters one, two, and three. The time shift is also recorded to be able to use with 
calculating the phase shift. The corner frequencies can be found using the data collected and the equation 
$ V_{out} = (7.07)   V_{in}$ . For the third filter two corner frequencies are found. The data is graphed on 
a logarithmic scaled graph.
%END KYLE

\section*{Results}

\section*{Conclusion}

\section*{Study Questions}

\end{document}
